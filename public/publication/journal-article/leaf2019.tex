@article{Leaf2019,
abstract = {A critical issue in understanding trophic connectivity in ecological systems is the lack in quality and quantity of information about feeding habits. In this work, we present a method for integrating a diversity of feeding habits data from published studies to evaluate the impact on indices that describe characteristics of individual taxa and their connectivity. We focus our study on feeding habits of the fishes of the northern Gulf of Mexico and seek to understand the importance of the forage fish Gulf Menhaden (Brevoortia patronus) in predator diets. We created a database of diet studies from the northern Gulf of Mexico that included six diet metrics: frequency of occurrence, wet weight, dry weight, number, volume, index of relative importance, and index of caloric importance. We then used this information to construct a set of traditional networks (all prey and predators were from a single taxonomic level and trophic connections were parameterized with a single diet metric). We also constructed a “robust” network where all taxa were identified to the lowest taxonomic level and trophic connections were parameterized using a resampling approach that included all available information. Linear regression and resampling methods were used to convert data reported in other diet metrics into the frequency of occurrence diet metric. For both traditional and robust networks, we used network indices to describe topological properties. With the robust network, we conducted removal simulations where the forage fish species Gulf Menhaden, and associated Clupeidae representatives, were removed from the network and the feeding effort of the predators was reallocated among their other prey items. We found that network and node-specific indices were sensitive to the choice of taxonomy and diet metric level. In the robust network, predator species with the greatest number of identified prey had the lowest precision in their connections and prey from the Arthropoda phyla had the lowest precision for connections. From the removal and reallocation simulations, we found that Actinopterygii and Arthropoda were the most impacted prey taxa with 1.2{\%} to 4.3{\%} increase in predation and approximately 23 taxa would receive 50{\%} of the reallocated predation. Overall, the resampling methods we present provide a potential means for combining disparate diet data and enables a comprehensive understanding of trophic interactions within an ecosystem.},
author = {Leaf, Robert T. and Oshima, Megumi C.},
doi = {10.1016/j.ecoinf.2018.12.005},
file = {:Users/MegumiOshima1/Downloads/1-s2.0-S157495411830195X-main.pdf:pdf},
issn = {15749541},
journal = {Ecological Informatics},
keywords = {Aggregation,Forage fish,Trophic dynamics,Weighted network},
number = {December 2018},
pages = {13--23},
publisher = {Elsevier},
title = {{Construction and evaluation of a robust trophic network model for the northern Gulf of Mexico ecosystem}},
url = {https://doi.org/10.1016/j.ecoinf.2018.12.005},
volume = {50},
year = {2019}
}
